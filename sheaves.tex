\chapter{Sheaves and Cohomology}

\section{Algebra Preliminaries}

We first provide some necessary background material on algebra.
All rings in this chapter will be assumed to be commutative and with unit.

\subsection*{Cochain Complex and Cohomology}

\begin{defn}
    A \emph{cochain complex} $C$ is a graded $R$-module $C=\bigoplus_{n\in\mathbb{N}}C^n$ with homomorphisms
    \[0\to C^0\xrightarrow{\delta^0}C^1\xrightarrow{\delta^1}\cdots\xrightarrow{\delta^{n-1}}C^n\xrightarrow{\delta^n}\cdots\]
    which satisfy $\delta^{n+1}\circ\delta^n=0$ for $n\geq 0$.
    A cochain map between chain complex is a graded homomorphism $f:C\to D$ of degree $0$, making the following diagram commute:
    \[\begin{tikzcd}
        0 \ar[r] & C^0 \ar[d, "f_0"] \ar[r,"\delta^0"] & C^1 \ar[d, "f_1"] \ar[r,"\delta^1"] & \cdots \ar[r,"\delta^{n-1}"] & C^n \ar[d, "f_n"] \ar[r,"\delta^n"] & \cdots \\
        0 \ar[r] & D^0 \ar[r,"d^0"] & D^1 \ar[r,"d^1"] & \cdots \ar[r, "d^{n-1}"] & D^n \ar[r, "d^n"] & \cdots
    \end{tikzcd}\]
\end{defn}

\begin{defn}
    The $i$th \emph{cohomology module} is defined by
    \[H^i(C)=\frac{\ker(\delta^i)}{\im(\delta^{i-1})}\]
    and the cohomology module is defined by the graded module $H^\bullet(C)=\bigoplus_{n\in\mathbb{N}}H^n(C)$.
    An element in $\ker(\delta^i)$ is called a \emph{cocycle}, and an element in $\im(\delta^{i-1})$ is called a \emph{coboundary}.
    A cochain map $f:C\to D$ induces a graded homomorphism $f^*:H^\bullet(C)\to H^\bullet(D)$ of degree $0$, i.e.\ for each $f_i:C^i\to D^i$, $f_i$ induces a homomorphism $f^*_i:H^i(C)\to H^i(D)$.
\end{defn}

We provide a lemma on homomorphism of cohomology module induced by cochain map.

\begin{lem}
    Let $C,D$ be cochain complex, $f,g:C\to D$ be cochain maps.
    Let $h:C\to D$ be a graded homomorphism of degree $-1$, then $\delta\circ h+h\circ\delta$ is a cochain map, and if $f-g=\delta\circ h+h\circ\delta$, then $f^*=g^*$.
\end{lem}

In such situation, $f$ and $g$ are called \emph{chain homotopic}.
\begin{proof}
    For the first claim, we can check
    \begin{align*}
        \delta\circ(\delta\circ h+h\circ\delta)&=\delta\circ h\circ\delta\\
        &=\delta\circ h\circ\delta+h\circ\delta\circ\delta\\
        &=(\delta\circ h+h\circ\delta)\circ\delta
    \end{align*}
    The second claim holds trivially.
\end{proof}

We are now going to proof the following theorem:
\begin{thm}[Zig-zag lemma]\label{zigzag}
    If the sequence of cochain complex
    \[0\to C\xrightarrow{f} D\xrightarrow{g} E\to 0\]
    is exact, then the long sequence of cohomology modules
    \[\begin{tikzcd}
        0 \ar[r] & H^0(C) \ar[r, "f_0^*"] & H^0(D) \ar[r, "g_0^*"] & H^0(E) \ar[dll, "\delta^*"] \\[3mm]
        {} & H^1(C) \ar[r, "f_1^*"] & H^1(D) \ar[r, "g_1^*"] & H^1(E) \ar[dll, "\delta^*"] \\[3mm]
        {} & \cdots & {} & {}
    \end{tikzcd}\]
    is exact.
\end{thm}

For this, we need the following lemmas.

\begin{lem}\label{coho short exact}
    Let $C$ be a cochain complex, the following sequence is exact for $n\geq 0$
    \[0\to\ H^n(C)\to\coker(\delta^{n-1})\xrightarrow{\delta^n}\ker(\delta^{n+1})\to H^{n+1}(C)\to 0\]
\end{lem}
\begin{proof}
    Clearly $\varphi^n:=\delta^n|_{\coker(\delta^{n-1})}$ is well-defined and $\ker(\varphi^n)=H^n(C),\coker(\varphi^n)=H^{n+1}(C)$.
\end{proof}

\begin{lem}[Snake lemma]
    Consider the following commutative diagram with exact rows:
    \[\begin{tikzcd}
        {} & X \ar[r, "f"] \ar[d,"\alpha"] & Y \ar[r, "g"] \ar[d,"\beta"] & Z \ar[r] \ar[d,"\gamma"] & 0 \\
        0 \ar[r] & U \ar[r, "f'"] & V \ar[r, "g'"] & W & {}
    \end{tikzcd}\]
    Then there exists a homomorphism $\delta:\ker(\gamma)\to\coker(\alpha)$ giving rise the following exact sequence
    \[\ker(\alpha)\to\ker(\beta)\to\ker(\gamma)\xrightarrow{\delta}\coker(\alpha)\to\coker(\beta)\to\coker(\gamma)\]
\end{lem}
\begin{proof}
    $\ker(\alpha)\to\ker(\beta)\to\ker(\gamma)$ is exact.
    Let $y\in\ker(g|_{\ker(\beta)})$, then $g(y)=0$, there exists an $x\in X$ such that $f(x)=y$.
    We must show $x\in\ker(\alpha)$.
    We have $0=\beta(y)=\beta(f(x))=f'(\alpha(x))$, and since $f'$ is injective, $\alpha(x)=0$, i.e.\ $x\in\ker(\alpha)$.
    Similarly $\coker(\alpha)\to\coker(\beta)\to\coker(\gamma)$ is exact.

    We now construct a $\delta$ making the sequence exact.
    Take $z\in\ker(\gamma)$, choose $y$ such that $g(y)=z$, then $g'(\beta(y))=\gamma(g(y))=\gamma(z)=0$, hence there exists a unique $u\in U$ such that $f'(u)=\beta(y)$.
    Set $\delta(z)=u+\im(\alpha)$.
    We show $\delta$ is well-defined.
    If $g(y')=z$, let $f'(u')=\beta(y')$, then $g(y-y')=0$, there exists an $x\in X$ such that $f(x)=y'-y$.
    Therefore
    \[f(u'-u)=\beta(y'-y)=\beta(f(x))=f'(\alpha(x))\]
    Since $f'$ in injective, we have $u'-u=\alpha(x)\in\im(\alpha)$, hence $\delta$ is well-defined.
    Consider $\ker(\beta)\to\ker(\gamma)\xrightarrow{\delta}\coker(\alpha)$, let $z\in\ker(\gamma)$ and $\delta(z)=0$.
    Then let $g(y)=z,f'(u)=\beta(y)$, we have $u\in\im(\alpha)$.
    Let $\alpha(x)=u$, then $\beta(y)=f'(\alpha(x))=\beta(f(x))$, we have $y-f(x)\in\ker(\beta)$.
    Moreover, $g(y-f(x))=z$, we have $\ker(\delta)\subset\im(g|_{\ker\beta})$, the sequence is exact.
    Similarly $\ker(\gamma)\xrightarrow{\delta}\coker(\alpha)\to\coker(\beta)$ is exact.
\end{proof}

\begin{proof}[Proof of Zig-zag lemma]
    Consider the following commutative diagram
    \[\begin{tikzcd}
        {} & 0 \ar[d] & 0 \ar[d] & 0 \ar[d] & {}\\
        {} & H^n(C) \ar[d] & H^n(D) \ar[d] & H^n(E) \ar[d] & {}\\
        {} & \coker(\delta^{n-1}_C) \ar[r, "f"] \ar[d] & \coker(\delta^{n-1}_D) \ar[r, "g"] \ar[d] & \coker(\delta^{n-1}_E) \ar[r] \ar[d] & 0 \\
        0 \ar[r] & \ker(\delta^n_C) \ar[r, "f"] \ar[d] & \ker(\delta^n_D) \ar[r, "g"] \ar[d] & \ker(\delta^n_E) \ar[d] & {} \\
        {} & H^{n+1}(C) \ar[d] & H^{n+1}(D) \ar[d] & H^{n+1}(E) \ar[d] & {}\\
        {} & 0 & 0 & 0 & {}
    \end{tikzcd}\]
    Diagram chasing shows the rows are exact, and by Lemma~\ref{coho short exact}, the columns are exact.
    Hence the result follows from Snake lemma.
\end{proof}

Our next result on cohomology is that cohomology is natural.

\begin{thm}[Naturality of cohomology]\label{natural}
    Let
    \[\begin{tikzcd}
        0\ar[r] & C \ar[r]\ar[d, "f"] & D \ar[r]\ar[d, "g"] & E \ar[r]\ar[d, "h"] & 0 \\
        0\ar[r] & C' \ar[r] & D'\ar[r] & E' \ar[r] & 0
    \end{tikzcd}\]
    be commutative diagram of cochain complexes with exact rows, then the long exact sequence of cohomology modules
    \[\begin{tikzcd}
        0 \ar[r] & H^0(C) \ar[r] \ar[d, "f^0"] & H^0(D) \ar[r] \ar[d, "g^0"] & H^0(E) \ar[r, "\delta^*"] \ar[d, "h^0"] & H^1(C) \ar[r] \ar[d, "f^1"] & \cdots\\
        0 \ar[r] & H^0(C') \ar[r] & H^0(D') \ar[r] & H^0(E') \ar[r, "(\delta')^*"] & H^1(C') \ar[r] & \cdots
    \end{tikzcd}\]
    is commute.
\end{thm}
\begin{proof}
    Diagram chasing.
\end{proof}

\subsection*{Direct Limit}

\begin{defn}
    A \emph{direct set} is a preordered set $(I,<)$ such that for $i,j\in I$, there is a $k\in I$ satisfying $i<k,j<k$.
    A \emph{direct system} is a set of rings or $R$-modules indexed by a direct set, with homomorphisms $\varphi_{ij}:M_i\to M_j$ whenever $i<j$ and $\varphi_{jk}\circ\varphi_{ij}=\varphi_{ik}$ whenever $i<j<k$.

    If $\{M_i\}_{i\in I}$ is a direct system, then the \emph{direct limit} of the direct system is a ring or $R$-module $\varinjlim_{i\in I}M_i$ with homomorphisms $\varphi_i:M_i\to\varinjlim_{i\in I}M_i$, satisfying for every $N$ with homomorphisms $\psi_i:M_i\to N$ that compatible with $\varphi_{ij}$'s, then there exists a unique homomorphism $\psi_N:\varinjlim_{i\in I}M_i\to N$ compatible with all homomorphisms.
\end{defn}

We just outline the construction of direct limits.
If $\{M_i\}_{i\in I}$ is a direct system, then $\varinjlim_{i\in I}M_i$ is a quotient of $\coprod_{i\in I}M_i$.
For rings, the coproduct notation stands for disjoint union, and quotient out the following equivalent relation: For $m_i\in M_i,m_j\in M_j$, $m_i\sim m_j$ if there exists $k\in I$ such that $i<k,j<k$ and $\varphi_{ik}(m_i)=\varphi_{jk}(m_j)$;
For $R$-modules, the coproduct notation stands for direct sum, and quotient out the submodule $Q$ generated by elements with form $\iota_i(m_i)-\iota_j(\varphi_{ij}(m_i))$, where $\iota_i$ is the natural embedding. 

\begin{prop}\label{colim exact}
    If $(I,<)$ is a direct set, $\{M_i'\}_{i\in I},\{M_i\}_{i\in I},\{M_i''\}_{i\in I}$ are direct systems, and for all $i\in I$ the sequence $M_i'\xrightarrow{f_i}M_i\xrightarrow{g_i}M_i''$ is exact, then the sequence
    \[\varinjlim_{i\in I}M_i'\xrightarrow{f}\varinjlim_{i\in I}M_i\xrightarrow{g}\varinjlim_{i\in I}M_i''\]
    is exact.
\end{prop}
\begin{proof}
    We prove for modules.
    Denote $H_i=\ker(g_i)/\im(f_i)$, $H=\ker(g)/\im(f)$.
    Then for each $I$, there is a canonical homomorphism $H_i\to H$.
    This gives rise a homomorphism $\varinjlim_{i\in I}H_i\to H$, we need to prove this homomorphism is surjective.
    Let $h\in H$, $h=m+\im(f)$, and let $m=\varphi_i(m_i)$ for some $m_i\in M_i$.
    Then there exists $j\geq i$ such that $\varphi_{ij}''(g_i(m_i))=0$, set $m_j=\varphi_{ij}(m_i)$, then $\varphi_j(m_j)=m$ and $m_j\in\ker(g_j)$.
    Hence the map is surjective.
    But $\varinjlim_{i\in I}H_i=0$, we obtain $H=0$.
\end{proof}

\begin{rem}
    In fact, further argument shows direct limit preserves homology.
    We refer to~\cite[Lemma 10.8.8]{stacks-project}.
\end{rem}

\section{Sheaves}

Instead of using \emph{espace \'{e}tale} as Morrow and Kodaira do, we shall adopt the standard definition of a sheaf.

\begin{defn}
    A \emph{presheaf} $\mathcal{F}$ on a topological space $X$ associates every open set to a group $\mathcal{F}(U)$, called the sections of $\mathcal{F}$ on $U$, and for each two open sets $U\subset V$ there is a \emph{restriction map} $\res^V_U:\mathcal{F}(V)\to\mathcal{F}(U)$ such that:
    \begin{enumerate}[(1)]
        \item For open sets $U\subset V\subset W$, we have $\res^V_U\circ\res^W_V=\res^W_U$;
        \item For open set $U$, we have $\res^U_U=\mathrm{id}_U$.
    \end{enumerate}
    A \emph{sheaf} $\mathcal{F}$ on $X$ is a presheaf satisfying the following two sheaf axioms:
    \begin{enumerate}[(1)]
        \item If for open set $U$ with open cover $U=\bigcup_{i\in I}U_i$ and $f\in\mathcal{F}(U)$, $\res^U_{U_i}f=0,\ \forall i\in I$, then $f=0$;
        \item If for open set $U$ with open cover $U=\bigcup_{i\in I}U_i$, and $f_i\in U_i$ with $\res^{U_i}_{U_i\cap U_j}f_i=\res^{U_j}_{U_i\cap U_j}f_j$, then there exists a unique $f\in U$ such that $\res^U_{U_i}f=f_i$.
    \end{enumerate}
    If the sections are rings, then the sheaf is called a \emph{sheaf of rings}, for $R$-modules \emph{mutatis mutandis}.
\end{defn}

\begin{eg}
    We give some examples of sheaves.
    \begin{enumerate}
        \item Let $M$ be a complex manifold.
        A \emph{holomorphic function} on $M$ is a complex valued function $f$ such that for every atlas $(U,\varphi)$ the function $f\circ\varphi^{-1}$ is holomorphic.
        Define $\mathcal{O}$ as for open set $U$ on $M$, $\mathcal{O}(U)$ is the $\mathbb{C}$-algebra of all holomorphic functions defined on $U$.
        (Notice that any open set on a complex manifold is a complex manifold.)
        \item Let $M$ be a differentiable manifold.
        Define $\mathcal{O}^*$ satisfy whose sections are nonzero holomorphic functions.
        \item Let $M$ be a complex manifold.
        Define $\mathcal{D}$ satisfy whose sections are differentiable functions.
        \item Let $M$ be a complex manifold.
        Define $\mathbb{Z},\mathbb{R},\mathbb{C}$ to be the sheaf with sections are locally constant $\mathbb{Z}$-, $\mathbb{R}$-, $\mathbb{C}$-valued functions.
    \end{enumerate}
\end{eg}

\begin{defn}
    Let $\mathcal{F}$ be a (pre)sheaf on $X$, $x\in X$.
    On the set of neighborhoods of $x$, we give a preorder as follows: $U\prec V$ if $V\subset U$, clearly this preorder makes the neighborhoods into a direct system.
    The direct limit $\varinjlim_{x\in U}\mathcal{F}_U$ is called the \emph{stalk} of $\mathcal{F}$ at $x$, denoted by $\mathcal{F}_x$.
    The elements in $\mathcal{F}_x$ are called \emph{germs}.
\end{defn}

\begin{prop}\label{inj of natural map}
    Let $\mathcal{F}$ be a sheaf on $X$, then for any $U\subset X$ open, the natural map
    \[
        \mathcal{F}(U)\to\prod_{x\in U}\mathcal{F}_x
    \]
    is injective.
\end{prop}
\begin{proof}
    We prove for sheaf of rings.
    Let $s\in\mathcal{F}(U)$ be in the kernel of the map, i.e. $(s_x)_{x\in U}=0$.
    Then by the construction of direct limit, for every $x\in U$, there is a neighborhood $U_x$ of $x$ such that $\res^U_{U_x}(s)=0$.
    But $\bigcup_{x\in U}U_x$ is an open cover of $U$, by sheaf axiom, we must have $s=0$.
    This proves the injectivity.
\end{proof}

\begin{defn}
    A \emph{morphism} $f:\mathcal{F}\to\mathcal{G}$ between (pre)sheaves on $X$ is a collection of homomorphisms $f(U)$ for open sets $U$ of $X$, satisfying for any $U\subset V$ the following diagram commutes
    \[\begin{tikzcd}
        \mathcal{F}(V)\ar[r,"f(V)"]\ar[d,"\res^V_U"] & \mathcal{G}(V)\ar[d,"\res^V_U"]\\
        \mathcal{F}(U)\ar[r,"f(U)"] & \mathcal{G}(U)
    \end{tikzcd}\]
\end{defn}

\begin{defn}
    Let $f:\mathcal{F}\to\mathcal{G}$ be a morphism of presheaves, then $f$ is injective if $f(U)$ is injective for each open set $U$.
\end{defn}

\begin{rem}
    As we shall see later, this is different from the injectivity of sheaf morphisms.
\end{rem}

\begin{defn}
    The \emph{kernel} of a sheaf morphism $f:\mathcal{F}\to\mathcal{G}$ is defined as $\ker(f)(U)=\ker(f(U))$.
    To avoid confusion, this means the module (or ring) of sections of kernel sheaf on $U$ is the kernel of $f(U)$.
    It's easy to verify this is a sheaf.
\end{defn}

\begin{eg}
    We cannot define the image of a morphism in the same way.
    For example, let's take the morphism
    \[\exp:\mathcal{O}\to\mathcal{O}^*\]
    Denote $\preim(\exp)(U)=\im(\exp(U))$, then for every simply connected open set $W\subset\mathbb{C}\setminus\{0\}$, $\mathrm{id}_W\in\preim(\exp)(U)$, but $\mathrm{id}_{\mathbb{C}\setminus\{0\}}\notin\preim(\exp)(\mathbb{C}\setminus\{0\})$, this shows $\preim(\exp)$ does not satisfy the sheaf axiom.
    To fix this, we need the following construction.
\end{eg}

\begin{defn}
    Let $\mathcal{F}$ be a presheaf, then there exists a sheaf $\mathcal{F}^+$ with presheaf morphism $\varphi:\mathcal{F}\to\mathcal{F}^+$, satisfying for any sheaf $\mathcal{G}$ and morphism $f:\mathcal{F}\to\mathcal{G}$, $f$ factors through $\varphi$, i.e.\ there exsits a unique $f^+:\mathcal{F}^+\to\mathcal{G}$ making the following diagram commute
    \[\begin{tikzcd}
        \mathcal{F} \ar[rr, "f"] \ar[dr, "\varphi"] & & \mathcal{G} \\
        {} & \mathcal{F}^+\ar[ur, "f^+"] & {}
    \end{tikzcd}\]
    The sheaf $\mathcal{F}^+$ is called the \emph{sheafification} of $\mathcal{F}$.
\end{defn}

By the definition, sheafification is unique up to an isomorphism (i.e.\ an invertible morphism).
\begin{cons}
    We now construct sheafification.
    Let $\mathcal{F}$ be a presheaf on $X$.
    Define a presheaf $\mathcal{F}^+$ as follows:
    For open set $U\subset X$, $\mathcal{F}^+(U)$ consists of maps $s:U\to\coprod_{x\in U}\mathcal{F}_x$,
    satisfying
    \begin{enumerate}[(1)]
        \item $s(x)\in\mathcal{F}_x$;
        \item For any $x\in U$, there exists a neighborhood $V$ of $x$ and $t\in\mathcal{F}(V)$, such that $s(y)=t_y$ for any $y\in V$, where $t_y$ is the image of natural map $\mathcal{F}(U)\to\mathcal{F}_y$. 
    \end{enumerate}
    Define
    \begin{align*}
        \varphi(U):\mathcal{F}(U)&\to\mathcal{F}^+(U)\\
        s&\mapsto(s_x)_{x\in U}
    \end{align*}
    Then it is easy to verify $\mathcal{F}^+$ is a sheaf, and $\mathcal{F}^+$ satisfy the universal property.
\end{cons}
For detailed construction and proof, we refer to~\cite[Section 6.17]{stacks-project}.

\begin{lem}\label{inj of sheafification map}
    The presheaf morphism $\varphi:\mathcal{F}\to\mathcal{F}^+$ is injective.
\end{lem}
\begin{proof}
    This follows from Proposition~\ref{inj of natural map}.
\end{proof}

\begin{defn}
    Let $f:\mathcal{F}\to\mathcal{G}$ be a sheaf morphism, and the sheaf $\preim(f)$ is defined as
    \[\preim(f)(U)=\im(f(U))\]
    Then the \emph{image} of $f$ is defined as the sheafification of $\preim(f)$, denoted by $\im(f)$.
\end{defn}

\begin{defn}
    Let $\mathcal{E}\xrightarrow{f}\mathcal{F}\xrightarrow{g}\mathcal{G}$ be sequence of sheaves and morphisms.
    The sequence is called \emph{exact}, if $\im(f)=\ker(g)$.
    
    A sheaf morphism $f:\mathcal{F}\to\mathcal{G}$ is called injective (resp.\ surjective) if the sequence $0\to\mathcal{F}\xrightarrow{f}\mathcal{G}$ (resp.\ $\mathcal{F}\xrightarrow{f}\mathcal{G}\to 0$) is exact.
\end{defn}

\begin{eg}
    The most fundamental exact sequence of sheaves is \emph{exponential sheaf sequence}:
    \[0\to\mathbb{Z}\to\mathcal{O}\xrightarrow{\exp}\mathcal{O}^*\to 0\]
    Where the first morphism is the obvious inclusion, and $\exp(f)=e^{2\pi\sqrt{-1}f}$.
    We check the surjectivity of last morphism.
    Let $f\in\mathcal{O}^*(U)$, then $U$ can be covered by simply connected open sets $\{U_i\}_{i\in I}$ (for instance, open disks).
    On $U_i$ we have $\mathcal{O}^*(U_i)=\im(\exp)(U_i)$, then there is $g_i\in\im(\exp)(U_i)$ such that $g_i=\res^U_{U_i}(f)$.
    By sheaf axiom, there is a unique $g\in\im(\exp)(U)$ such that $\res^U_{U_i}(g)=g_i$ for $i\in I$.
    Mapping $f\mapsto g$, we get a sheaf morphism $\mathcal{O}^*\to\im(\exp)$, then by the uniqueness of sheafification, $\mathcal{O}^*=\im(\exp)$.
\end{eg}

We have a simpler criterion for exactness.
\begin{prop}\label{exactness of sheaves}
    The sequence $\mathcal{E}\xrightarrow{f}\mathcal{F}\xrightarrow{g}\mathcal{G}$ of sheaves on $X$ is exact if and only if $\mathcal{E}_x\xrightarrow{f_x}\mathcal{F}_x\xrightarrow{g_x}\mathcal{G}_x$ is exact for each $x\in X$.
\end{prop}
\begin{proof}
    One side is easy.
    We suppose $\mathcal{E}_x\xrightarrow{f_x}\mathcal{F}_x\xrightarrow{g_x}\mathcal{G}_x$ is exact for each $x\in X$, and prove $\mathcal{E}\xrightarrow{f}\mathcal{F}\xrightarrow{g}\mathcal{G}$ is exact.
    Let $U\subset X$ be open.
    Assume $s\in\im(f)(U)$, substitute $\mathcal{G}$ by its sheafification, we have
    \[g(s)=(g_x(s_x))_{x\in U}\]
    But $s_x\in\im(f)_x=\ker(g)_x$, we have $g(s)=0$, i.e.\ $\im(f)(U)\subset\ker(g)(U)$.
    Conversely, if $g(s)=0$, for each $x\in U$ we have $g_x(s_x)=0$.
    Then $s_x\in\im(f)_x=\preim(f)_x$, we show that $s\in\im(f)(U)$.
    For each $x\in U$, there is a neighborhood $U_x$ of $x$ such that there exists a $t\in\mathcal{F}(U)$ satisfying $s_y=t_y$ for every $y\in U_x$.
    We must show that $t\in\preim(f)(U_x)$.
    Since $s_x\in\preim(f)_x$, there is a neighborhood $V$ of $x$ such that there exists an $r\in\im(f(V))$ satisfying $r_x=s_x$.
    Then $r_x=t_x$, by shrinking $U_x$ we can assume $r$ and $t$ coincide on $U_x$.
    Hence $t\in\preim(f)(U_x)$, this shows $s\in\im(f)(U)$.
\end{proof}

\section{Cohomology Groups}

In this section we define the \emph{\v{C}ech cohomology groups} of a paracompact Hausdorff space.
We shall write $f|_U$ instead of $\res^V_U(f)$ for short.

Fix a sheaf $\mathcal{F}$ of $R$-modules.
Let $\underline{U}=\{U_i:\ i\in I\}$ be a locally finite open cover, define
\[C^q(\underline{U},\mathcal{F})=\prod_{(i_0,\cdots,i_q)\in I^{q+1}}\mathcal{F}(U_{i_0\cdots i_q})\]
Where $U_{i_0\cdots i_q}=U_{i_0}\cap U_{i_1}\cap\cdots\cap U_{i_q}$, and $c\in C^q$ is denoted by $c=\left(c_{(i_0,\cdots,i_q)}\right)$.
The set $C^q(\underline{U},\mathcal{F})$ has a natural $R$-module structure inherited from the sections of sheaf $\mathcal{F}$.

Moreover, we define a coboundary operator
\[(\delta^q(c))_{(i_0,\cdots,i_{q+1})}=\sum_{j=0}^{q+1}(-1)^j\left.c_{(i_0,\cdots,\widehat{i_j},\cdots,i_{q+1})}\right|_{U_{i_0\cdots i_{q+1}}}\]
It's clear that $\delta^{q+1}\circ\delta^q=0$, this makes $C(\underline{U},\mathcal{F}):=\bigoplus_{q\geq 0}C^q(\underline{U},\mathcal{F})$ into a cochain complex, called \emph{\v{C}ech complex}.
Thus we define

\begin{defn}
    The \emph{\v{C}ech cohomology group} related to $\underline{U}$ is the cohomology $\mathbb{R}$-module of \v{C}ech complex $C(\underline{U},\mathcal{F})$, denoted by $H^\bullet(\underline{U},\mathcal{F})$.
\end{defn}

The cohomology group is an $R$-module, but we still use the traditional name group.

We now construct the \v{C}ech cohomology group that does not depend on open cover.
We define a preorder on the collection of all locally finite open cover as follows:
Let $\underline{U}=\{U_i\}_{i\in I},\underline{V}=\{V_j\}_{j\in J}$ be two locally finite open cover, define $\underline{U}\prec\underline{V}$ if there exists a map $\rho:J\to I$ such that $V_j\subset U_{\rho(j)}$ for each $j\in J$.
We call $\rho$ a \emph{refinement map} of $\underline{U}$.
Then we can define a cochain map (with abuse of notation) $\rho:C(\underline{U},\mathcal{F})\to C(\underline{V},\mathcal{F})$ by
\[(\rho(c))_{(i_0,\cdots,i_q)}=c_{(\rho(i_0),\cdots,\rho(i_q))}|_{V_{i_0\cdots i_q}}\]
The commutative property $\rho\circ\delta=\delta\circ\rho$ is immediate.
Then a refinement map induces a homomorphism between \v{C}ech cohomology groups $H^\bullet(\underline{U},\mathcal{F})\to H^\bullet(\underline{V},\mathcal{F})$.
If $\rho':J\to I$ is another refinement map, then $\rho,\rho'$ are chain homotopic:
To see this, we define a map $h^q:C^q(\underline{U},\mathcal{F})\to C^{q-1}(\underline{V},\mathcal{F})$ as
\[h^q(c)_{(i_0,\cdots,i_q)}=\sum_{j=0}^{q-1}(-1)^j\left.c_{(\rho(i_0),\cdots,\rho(j),\rho'(j+1),\cdots,\rho'(q-1))}\right|_{V_{(i_0,\cdots,i_{q-1})}}\]
It's straightforward to check $\delta^{q-1}\circ h^q+h^{q+1}\circ\delta^q=(\rho')^q-\rho^q$, then the homomorphism $\rho:H^\bullet(\underline{U},\mathcal{F})\to H^\bullet(\underline{V},\mathcal{F})$ only depends on the open cover.

On a paracompact Hausdorff space, two locally finite open cover has a common refinement.
This makes $H^\bullet(\underline{U},\mathcal{F})$ into a direct set.
Thus we define
\begin{defn}
    The \emph{\v{C}ech cohomology group} of a paracompact Hausdorff space $X$ is 
    \[H^\bullet(X,\mathcal{F}):=\varinjlim_{\underline{U}}H^\bullet(\underline{U},\mathcal{F})\]
    We also denote \v{C}ech cohomology group by $\check{H}^\bullet(X,\mathcal{F})$ if there is any confusion.
\end{defn}

We describe some low order cohomology groups.

\begin{prop}
    $H^0(X,\mathcal{F})\cong\mathcal{F}(X)$.
\end{prop}
\begin{proof}
    Let $\underline{U}$ be a locally finite open cover.
    Then $H^0(\underline{U},\mathcal{F})=\ker(\delta^0)$.
    Let $(\sigma_i)\in\ker(\delta^0)$, then $\sigma_i|_{U_{ij}}-\sigma_j|_{U_{ij}}=0$.
    By sheaf axiom, there exists a unique $\sigma\in\mathcal{F}(X)$ such that $\sigma|_{U_i}=\sigma_i$.
    If given a $\tau\in\mathcal{F}(X)$, then $(\tau|_{U_i})\in\ker(\delta^0)$.
    Hence $H^0(\underline{U},\mathcal{F})\cong\mathcal{F}(X)$, and therefore $H^0(X,\mathcal{F})\cong\mathcal{F}(X)$.
\end{proof}

\begin{prop}\label{H^1(U) to H^1(X)}
    The natural map $H^1(\underline{U},\mathcal{F})\to H^1(X,\mathcal{F})$ is injective.
\end{prop}
\begin{proof}
    Suppose $\sigma\in H^1(\underline{U},\mathcal{F})$.
    Then $\sigma=(\sigma_{(i_0,i_1)})+\im(\delta^0)$, $\sigma_{(i_0,i_1)}\in\mathcal{F}(U_{i_0i_1})$ and $\delta(\sigma)=(\sigma_{(i_1,i_2)}-\sigma_{(i_0,i_2)}+\sigma_{(i_0,i_1)})=0$.
    By taking $i_0=i_1=i$, we obtain $\sigma_{(i,i)}=0$.
    Assume $\sigma$ is in the kernel of natural map, then $\rho(\sigma)\in\im(\delta^0)$ for some open cover $\underline{V}$ such that $\underline{U}\prec\underline{V}$ with $\rho:J\to I$. 
    We can assume that $U_i=\bigcup_{\rho(j)=i}V_j$.
    For if not, let $\underline{W}:=\{W_{ij}=U_i\cap V_j\}$.
    Then $\underline{W}$ is a refinement of $\underline{V}$ with refinement map $\varpi:I\times J\to J$ and $\varpi\circ\rho(\sigma)\in\im(\delta^0)$.
    Since $\varpi$ between cohomology groups is independent of the choice of refinement map, we may choose $\rho:I\times J\to I$ be the projection map and replace $\underline{V}$ with $\underline{W}$.
    Now denote $\rho(\sigma)=\{\tau_{(j_0,j_1)}\}$ where $\tau_{(j_0,j_1)}=\sigma_{(\rho(j_0),\rho(j_1))}|_{V_{j_0j_1}}$.
    Then there exists $(\tau_j)\in C^0(\underline{V},\mathcal{F})$ satisfying $(\tau_{(j_0,j_1)})=\delta(\tau_j)=(\tau_{j_1}-\tau_{j_0})$.
    If $\rho(j_0)=\rho(j_1)$, then $(\tau_{j_1}-\tau_{j_0})|_{V_{j_0j_1}}=\tau_{(j_0,j_1)}=\sigma_{(\rho(j_0),\rho(j_1))}=0$.
    So we can glue $\{\tau_j\}_{\rho(j)=i}$ to get a section $\sigma_i\in\mathcal{F}(U_i)$ since $U_i=\bigcup_{\rho(j)=i}V_j$.
    Since $U_{i_0i_1}=\bigcup_{\rho(j_0)=i_0,\rho(j_1)=j_1}V_{j_0j_1}$ and 
    \[(\sigma_{\rho(j_1)}-\sigma_{\rho(j_0)})|_{V_{j_0j_1}}=(\tau_{j_1}-\tau_{j_0})|_{V_{j_0j_1}}=\tau_{(j_0,j_1)}=\sigma_{(\rho(j_0),\rho(j_1))}|_{V_{j_0j_1}}\]
    we obtain $\delta(\sigma_i)=((\sigma_{i_1}-\sigma_{i_0})|_{U_{i_0i_1}})=(\sigma_{(i_0,i_1)})$, that is, $\sigma+\im(\delta^0)=0$.
    Hence the kernel is zero, the natural map is injective.
\end{proof}

Next we discuss the exact sequence of cohomology groups.

\begin{thm}\label{exact cohomology groups}
    Assume the sequence of sheaves on paracompact Hausdorff space $X$
    \[0\to\mathcal{E}\xrightarrow{f}\mathcal{F}\xrightarrow{g}\mathcal{G}\to 0\]
    is exact, then the sequence of cohomology groups
    \[\begin{tikzcd}
        0 \ar[r] & H^0(X,\mathcal{E}) \ar[r, "f_0^*"] & H^0(X,\mathcal{F}) \ar[r, "g_0^*"] & H^0(X,\mathcal{G}) \ar[dll, "\delta^*"] \\[3mm]
        {} & H^1(X,\mathcal{E}) \ar[r, "f_1^*"] & H^1(X,\mathcal{F}) \ar[r, "g_1^*"] & H^1(X,\mathcal{G}) \ar[dll, "\delta^*"] \\[3mm]
        {} & \cdots & {} & {}
    \end{tikzcd}\]
    is exact.
\end{thm}
\begin{proof}
    Let $\underline{U}$ be a locally finite open cover.
    A morphism of sheaves $f:\mathcal{E}\to\mathcal{F}$ gives rise to a cochain map between \v{C}ech complexes $f^\sharp:C(\underline{U},\mathcal{E})\to C(\underline{U},\mathcal{F})$ by sending $\sigma_{(i_0,\cdots,i_n)}$ to $f(\sigma_{(i_0,\cdots,i_n)})$.
    Clearly this commutes with coboundary operator, and it is easy to see the correspondence is left-exact, i.e.\
    \[0\to C(\underline{U},\mathcal{E})\xrightarrow{f^\sharp}C(\underline{U},\mathcal{F})\xrightarrow{g^\sharp}C(\underline{U},\mathcal{G})\]
    is exact.
    Let the image of $g^\sharp$ be $C_0(\underline{U},\mathcal{G})$, and the cohomology group of this complex be $H^\bullet_0(\underline{U},\mathcal{G})$.
    By Zig-zag Lemma (Theorem~\ref{zigzag}), we have a long exact sequence of
    \[\cdots\to H^n(\underline{U},\mathcal{E})\to H^n(\underline{U},\mathcal{F})\to H^n_0(\underline{U},\mathcal{G})\to\cdots\]
    By Proposition~\ref{colim exact}, direct limit preserves exactness, so taking direct limit, we obtain a long exact sequence
    \[\cdots\to H^n(X,\mathcal{E})\to H^n(X,\mathcal{F})\to H^n_0(X,\mathcal{G})\to\cdots\]
    By~\cite[25.\ Proposition 7]{FAC}, $H^n_0(X,\mathcal{G})\xrightarrow{\sim}H^n(X,\mathcal{G})$ on a paracompact Hausdorff space, hence we obtain the required long exact sequence.
\end{proof}

\begin{thm}
    Let
    \[\begin{tikzcd}
        0\ar[r] & \mathcal{E} \ar[r]\ar[d, "f"] & \mathcal{F} \ar[r]\ar[d, "g"] & \mathcal{G} \ar[r]\ar[d, "h"] & 0 \\
        0\ar[r] & \mathcal{E}' \ar[r] & \mathcal{F}'\ar[r] & \mathcal{G}' \ar[r] & 0
    \end{tikzcd}\]
    be commutative diagram of sheaves with exact rows, then the long exact sequence of cohomology groups
    \[\begin{tikzcd}
        0 \ar[r] & H^0(X,\mathcal{E}) \ar[r] \ar[d, "f^0"] & H^0(X,\mathcal{F}) \ar[r] \ar[d, "g^0"] & H^0(X,\mathcal{G}) \ar[r, "\delta^*"] \ar[d, "h^0"] & \cdots\\
        0 \ar[r] & H^0(X,\mathcal{E}') \ar[r] & H^0(X,\mathcal{F}') \ar[r] & H^0(X,\mathcal{G}') \ar[r, "(\delta')^*"] & \cdots
    \end{tikzcd}\]
    is commute.
\end{thm}
\begin{proof}
    Use Theorem~\ref{natural}.
\end{proof}

Finally we give a brief discussion on fine sheaves.

\begin{defn}
    $\mathcal{F}$ is a \emph{fine sheaf}, if for any locally finite open cover $\{U_i\}_{i\in I}$ of $X$, there exists a set $\{h_i\}_{i\in I}$ of morphisms $h_i:\mathcal{F}\to\mathcal{F}$ such that
    \begin{enumerate}[(1)]
        \item $h_i(\mathcal{F}_x)=0$ for $x\notin\overline{W}_i$, where $\overline{W}_i\subset U_i$ is a closed sub set of $U_i$;
        \item $\sum_ih_i=\mathrm{id}$.
    \end{enumerate}
\end{defn}

% We clarify the last summation.
% Let $x\in X$, then there is a neighborhood $V_x$ of $x$ intersects finite number of $\overline{W}_i$'s.
% Then on each $\mathcal{F}_y$ for $y\in V_x$, $\sum_i h_i$ is a finite sum.
% Thus we have a finite sum $\sum_i h_i:\prod_{y\in V_x}\mathcal{F}_y\to\prod_{y\in V_x}\mathcal{F}_y$.
% By Proposition~\ref{inj of natural map}, $\mathcal{F}(V_x)$ is a subset of $\prod_{y\in V_x}\mathcal{F}_y$, hence by restriction we define a finite sum $\sum_{i}h_i:\mathcal{F}(V_x)\to\prod_{y\in V_x}\mathcal{F}_y$.
% Since each $h_i$ takes sections of $\mathcal{F}(V_x)$ to sections, $\sum_{i}h_i$ maps $\mathcal{F}(V_x)$ to itself.
% Hence $\sum_{i}h_i$ is well-defined on $V_x$.
% Since $\{V_x\}$ covers $X$, we can extend the sum onto the whole sheaf by gluing.

\begin{eg}
    Let $\mathcal{D}$ be the sheaf of differentiable functions on a differentiable manifold $M$.
    Given a locally finite open cover $\{U_i\}$, we have a \emph{partition of unity} subordinate to $\{U_i\}$, that is, a set $\{\rho_i\}$ of differentiable functions on $M$ such that
    \begin{enumerate}[(1)]
        \item $\rho_i(x)=0$ for $x\notin\overline{W}_i$;
        \item $\sum\rho_i=1$.
    \end{enumerate}
    Then for any differentiable function $f$ on $M$, define $h_i(f)=\rho(x)f(x)$, then $h_i$ induces a morphism $\mathcal{D}\to\mathcal{D}$.
    These $h_i$'s show that $\mathcal{D}$ is fine.
\end{eg}

\begin{thm}
    If $\mathcal{F}$ is fine, then $\mathcal{F}$ is acyclic, i.e.\ $H^n(X,\mathcal{F})=0$ for $n>0$. 
\end{thm}
\begin{proof}
    Given a locally finite open cover $\{U_\beta\}$, we show that $H^n(\{U_\beta\},\mathcal{F})=0$ for $n>0$.
    Let $(\sigma_{(i_0,\cdots,i_n)})\in\ker\delta^n$, we need to show that $(\sigma_{(i_0,\cdots,i_n)})\in\im\delta^{n-1}$.
    Let
    \[\tau_{(i_0,\cdots,i_{n-1})}=\sum_{\beta}h_\beta\sigma_{(\beta,i_0,\cdots,i_{n-1})}\]
    Where $h_\beta\sigma_{(\beta,i_0,\cdots,i_{n-1})}$ can be extended to $U_{i_0\cdots i_{n-1}}$ since $h_\beta$ is supported on $U_\beta$.
    We compute (omit restriction symbol)
    \begin{align*}
        (\delta(\tau))_{(i_0,\cdots,i_n)}&=\sum_{j=0}^n(-1)^j\tau_{(i_0,\cdots,\widehat{i_j},\cdots,i_n)}\\
        &=\sum_{j=0}^n(-1)^j\sum_\beta h_\beta\sigma_{(\beta,i_0,\cdots,\widehat{i_j},\cdots,i_n)}\\
        &=\sum_\beta h_\beta\sum_{j=0}^n(-1)^j\sigma_{(\beta,i_0,\cdots,\widehat{i_j},\cdots,i_n)}\\
        &=\sum_\beta h_\beta(\sigma_{(i_0,\cdots,i_n)}-(\delta(\sigma))_{(\beta,i_0,\cdots,i_n)})\\
        &=\sigma_{(i_0,\cdots,i_n)}
    \end{align*}
    Hence $(\sigma_{(i_0,\cdots,i_n)})\in\im\delta^{n-1}$, and therefore $H^n(\{U_\beta\},\mathcal{F})=0$.
    Thus $H^n(X,\mathcal{F})=0$ for $n>0$.
\end{proof}

\section{Vector Bundles}

We define complex analytic and differentiable vector bundles.
Let $M$ be a complex or differentiable manifold.

\begin{defn}
    A \emph{complex analytic} (resp.\ \emph{differentiable}) \emph{vector bundle} over $M$ is a complex (resp.\ differentiable) manifold $F$ together with a holomorphic (resp.\ differentiable) map $\pi:F\to M$ onto $M$ such that, for a sufficiently fine locally finite open cover $\underline{U}=\{U_j\}$ of $M$:
    \begin{enumerate}[(1)]
        \item There is a complex analytic (resp.\ differentiable) equivalence $f_j$ from $\pi^{-1}(U_j)$ to $U_j\times\mathbb{C}^n$ (resp.\ $U_j\times\mathbb{R}^n$) making the following diagram commute
        \[\begin{tikzcd}
            \pi^{-1}(U_j) \ar[d,"\pi"]\ar[r,"f_j"] & U_j\times\mathbb{C}^n \ar[d,"\pi_j"]\\
            U_j \ar[r,"\mathrm{id}"] & U_j
        \end{tikzcd}\]
        where $\pi_j(z_j,\zeta)=z_j$.
        For differentiable vector bundle we have similar diagram.
        \item Each transition function $f_{jk}=f_j\circ f_k^{-1}$ is holomorphic (resp.\ differentiable), and its restriction on each $\pi^{-1}(z)$ for $z\in U_j\cap U_k$ is a linear transformation, that is, a holomorphic matrix-valued function.
    \end{enumerate}
    We call each $\pi^{-1}(z)$ a \emph{fiber}, and the number $n$ the \emph{rank} of the bundle.
    The cover $\underline{U}$ is called a \emph{trivializing cover}.
\end{defn}

\begin{rem}
    A vector bundle is completely determined by the transition functions.
    That is, given a locally finite cover $\{U_j\}$ of $M$ and maps $f_{ij}:U_i\cap U_j\to\mathrm{GL}_n(\mathbb{C})$ (resp.\ $f_{ij}:U_i\cap U_j\to\mathrm{GL}_n(\mathbb{R})$) satisfying the following equalities
    \begin{equation}
        \begin{gathered}
            f_{ij}\cdot f_{ji}=I \\
            f_{ij}\cdot f_{jk}\cdot f_{ki}=I
        \end{gathered}\label{transition functions}
    \end{equation}
    for indices $i,j,k$, then there exists a unique complex analytic (resp.\ differentiable) vector bundle on $M$.
    (cf.\ \cite[p.\ 66]{G-H})
\end{rem}

\begin{defn}
    We say that $F$ and $F'$ are holomorphically (resp.\ differentiably) equivalent if there is a biholomorphic (resp.\ bidifferentiable) map $\varphi:F'\to F$ such that
    \begin{enumerate}[(1)]
        \item The diagram
        \[\begin{tikzcd}
            F\ar[rr,"\varphi"]\ar[dr,"\pi"'] & & F'\ar[dl,"\pi'"] \\
            & M &
        \end{tikzcd}\]
        is commute.
        \item On each fiber $\varphi$ is a linear transformation, that is, if $\{U_j\}$ is a trivializing cover, then there is a holomorphic (resp.\ differentiable) matrix-valued function $h_j$ on each $U_j$ such that $f_j'\circ\varphi\circ f_j^{-1}=h_j$.
    \end{enumerate}
\end{defn}

We list here some methods to construct new bundles from old bundles.
Let $F,G$ be complex analytic or differentiable vector bundles of rank $n,m$ respectively, and $\{U_j\}$ a trivializing cover with $F=\{f_{ij}\},G=\{g_{ij}\}$.
We define the following new objects:
\begin{enumerate}[(1)]
    \item \emph{Whitney Sum} $F\oplus G$:
    This is a bundle of rank $n+m$ defined by transition functions
    \[h_{ij}=\begin{bmatrix}
        f_{ij} & 0 \\
        0 & g_{ij}
    \end{bmatrix}\]
    \item \emph{Tensor Product} $F\otimes G$:
    This is a bundle of rank $nm$ defined by transition functions $\{f_{ij}\otimes g_{ij}\}$.
    Recall the \emph{Kronecker product} of two matrices $A=[a_{ij}]_{1\leq i,j\leq n}$ and $B$ is a block matrix
    \[A\otimes B=\begin{bmatrix}
        a_{11}B & a_{12}B & \cdots & a_{1n}B\\
        a_{21}B & a_{22}B & \cdots & a_{2n}B\\
        \vdots & \vdots & \ddots & \vdots \\
        a_{n1}B & a_{n2}B & \cdots & a_{nn}B
    \end{bmatrix}\]
    \item \emph{Dual} bundle $F^*$ of $F$:
    This is a bundle of rank $n$ defined by transition functions $f_{ij}^*=(f_{ij}^{-1})^\top=f_{ji}^\top$,
    the transpose inverse of $f_{ij}$.
    \item \emph{Complex Conjugate} $\overline{F}$ of $F$:
    This is a bundle of rank $n$ defined by transition functions $\overline{f_{ij}}$.
\end{enumerate}

We now define subbundles and quotient bundles.
Suppose by a suitable choice of trivializing cover and coordinates, the transition functions $\{f_{ij}\}$ can be written as follows:
\[f_{ij}=\begin{bmatrix}
    A_{ij} & B_{ij}\\
    0 & C_{ij}
\end{bmatrix}\]
where $A_{ij}$ are $m\times m\ (m<n)$ matrices.
Then by the multiplication of block matrices, $\{A_{ij}\}$ satisfy the identity~\eqref{transition functions}, hence it determines a bundle $F'$.
We call $F'$ a \emph{subbundle} of $F$.
Similarly $\{C_{ij}\}$ satisfy the identity~\eqref{transition functions}, we call it determines the \emph{quotient bundle} $F''=F/F'$.

Next we discuss line bundles and Chern class.
\begin{defn}
    A \emph{line bundle} on complex manifold $M$ is a complex analytic bundle $L$ of rank $1$.
\end{defn}

\begin{prop}\label{picard}
    The tensor product is well-defined on the equivalent classes of line bundles on $M$, and makes it into a group.
\end{prop}
\begin{proof}
    Notice that dual gives the inverse of (the equivalent class of) a bundle, and everything is clear.
\end{proof}

\begin{defn}
    The group in Proposition~\ref{picard}~is called the \emph{Picard group} of $M$, denoted by $\Pic(M)$.
\end{defn}

\begin{prop}
    On a complex manifold $M$ we have $\Pic(M)\cong H^1(M,\mathcal{O}^*)$.
\end{prop}
\begin{proof}
    Let $\{l_{ij}\}_{\underline{U}}$ be a line bundle.
    Consider $l_{ij}$ as an element of $\mathcal{O}^*(U_i\cap U_j)$.
    On $U_i\cap U_j\cap U_k$, by~\eqref{transition functions}~we have
    \begin{gather*}
        l_{ij}\cdot l_{jk}\cdot l_{ki}=1\\
        \iff l_{jk}\cdot l_{ik}^{-1}\cdot l_{ij}=1
    \end{gather*}
    Hence $(l_{ij})\in\ker\delta^1$.
    If two line bundles $(l_{ij})$ and $(l'_{ij})$ are equivalent, there are nonvanishing functions $h_j$ on $U_j$ such that
    \[(l_j')^{-1}\cdot h_j\cdot l_j=(l_k')^{-1}\cdot h_k\cdot l_k\]
    That is $l_{jk}=l'_{jk}\cdot h_k\cdot h_j^{-1}$, then $(l_{jk})$ and $(l'_{jk})$ are cohomologous.
    Hence an equivalent class defines an element in $H^1(\underline{{U}},\mathcal{O}^*)$, and by Proposition~\ref{H^1(U) to H^1(X)}, it actually defines an element in $H^1(M,\mathcal{O}^*)$.
    It is clear that this correspondence is a homomorphism.
    Conversely, it is easy to construct a line bundle from an element of $H^1(M,\mathcal{O}^*)$.
    This proves our proposition.
\end{proof}

\begin{rem}
    Same argument shows the group of differentiable equivalent classes of differentiable line bundles is isomorphic to $H^1(M,\mathcal{D}^*)$.
\end{rem}

We now construct Chern class, an important invariant of line bundles.
We have the exponential sequence for $\mathcal{O}$ and $\mathcal{D}$:
\[\begin{tikzcd}
    0 \ar[r] & \mathbb{Z} \ar[d, "\mathrm{id}"] \ar[r] & \mathcal{O} \ar[d, hook] \ar[r, "\exp"] & \mathcal{O}^* \ar[d, hook] \ar[r] & 0\\
    0 \ar[r] & \mathbb{Z} \ar[r] & \mathcal{D} \ar[r, "\exp"] & \mathcal{D}^* \ar[r] & 0
\end{tikzcd}\]
with exact rows.
This yields a commutative diagram of cohomology groups with exact rows
\[\begin{tikzcd}
    H^1(M,\mathcal{O}) \ar[r] \ar[d] & H^1(M,\mathcal{O}^*) \ar[r, "\delta^*"] \ar[d, "i"] & H^2(M,\mathbb{Z}) \ar[r] \ar[d] & H^2(M,\mathcal{O}) \ar[d]\\
    H^1(M,\mathcal{D}) \ar[r] & H^1(M,\mathcal{D}^*) \ar[r, "\delta^*"] & H^2(M,\mathbb{Z}) \ar[r] & H^2(M,\mathcal{D})
\end{tikzcd}\]

\begin{defn}
    The (first) \emph{Chern class} $c$ is defined as $\delta^*:\Pic(M)\to H^2(M,\mathbb{Z})$.
\end{defn}

\begin{prop}
    The Chern class $c(L)$ of a complex analytic line bundle $L$ represents the differentiable equivalence class of $L$.
\end{prop}
\begin{proof}
    Since $\mathcal{D}$ is a fine sheaf, $\mathcal{D}$ is acyclic.
    Hence we have $0\to H^1(M,\mathcal{D}^*)\xrightarrow{\delta^*} H^2(M,\mathbb{Z})\to 0$, that is, $H^1(M,\mathcal{D}^*)\cong H^2(M,\mathbb{Z})$, and the previous one is the group of differentiable equivalent classes of differentiable line bundles on $M$.
\end{proof}

Let us give an explicit description of $c(L)$.
If $L=\{l_{ij}\}$, $l_{ij}\cdot l_{jk}\cdot l_{ki}=1$, and
\[\log{l_{ij}}+\log{l_{jk}}+\log{l_{ki}}=2\pi\sqrt{-1}c_{ijk}\]
Then $c(L)=(c_{ijk})\in H^2(M,\mathbb{Z})$.
It easy to see different analytic branches gives cohomologous $(c_{ijk})$.

We now discuss sections of vector bundles.
\begin{defn}
    A holomorphic (resp.\ differentiable) \emph{section} of $F$ over $U\subset M$ is a holomorphic (resp.\ differentiable) map $\varphi:U\to F$ such that $\pi\circ\varphi=\mathrm{id}_U$.
    Denote the $\mathbb{C}$- (resp.\ $\mathbb{R}$-) vector space of sections by $\Gamma(U,F)$.

    We define a presheaf $\mathcal{O}(F)$ by setting $\mathcal{O}(F)(U)=\Gamma(U,F)$, and it is easy to check this is a sheaf, called the \emph{sheaf of holomorphic sections of} $F$.
    Similarly we have a \emph{sheaf of differentiable sections of} $F$ denoted by $\mathcal{D}(F)$.
\end{defn}

Locally $\mathcal{O}(F)|_U=\mathcal{O}|_U\oplus\cdots\oplus\mathcal{O}|_U$ ($n$ times) on a trivializing open set $U$.

We give the definition of various tangent bundles and vector fields.
\footnote{
    The reader can notice that Morrow and Kodaira use subscript to denote the indices of components in Chapter 1.
    However, from now on they use superscript to denote the indices of components, we shall keep the same notation with the book.
}
\begin{defn}
    Let $(U_\alpha,z_\alpha)_{\alpha\in A}$ be a locally finite atlas of $M$, the \emph{holomorphic tangent bundle} $T(M)$ is defined by Jacobi matrices of $z_\alpha\circ(z_\beta)^{-1}$.
    The conjugate $\overline{T(M)}$ of $T(M)$ is called \emph{conjugate tangent bundle} of $M$, and the sum $\mathcal{T}(M):=T(M)\oplus\overline{T(M)}$ is called \emph{complexified tangent bundle} of $M$.
\end{defn}

We note that on a chart $(U,\zeta)$, the fiber on a point $z$ is isomorphic to $\langle\partial/\partial{\zeta^1},\cdots,\partial/\partial{\zeta^n}\rangle_\mathbb{C}$.
If $(V,\xi)$ is another chart, chain rule shows the transition function is exactly the Jacobi matrix of $\xi\circ\zeta^{-1}$.
Hence we have a well-defined tangent space.
\begin{defn}
    The \emph{tangent space} at point $z$ is given by $\langle\partial/\partial{\zeta_1},\cdots,\partial/\partial{\zeta_n}\rangle_\mathbb{C}$ for a chart $(U,\zeta)$, denoted by $T_z(M)$.
\end{defn}

\begin{rem}
    If
    \[\mathcal{T}_\mathbb{R}(M):=\coprod_{z\in M}\mathcal{T}_{\mathbb{R},x}(M)\]
    is the real tangent bundle of $M$ as a differentiable manifold, then $\mathcal{T}(M)=\mathcal{T}_\mathbb{R}(M)\otimes\mathbb{C}$, where $\mathbb{C}$ is the trivial complex analytic line bundle on $M$ considered as a differentiable bundle.
\end{rem}

\begin{defn}
    A \emph{holomorphic vector field} is a holomorphic section of $T(M)$.
    The sheaf of holomorphic vector fields is denoted by $\Theta$. 
\end{defn}

By linear algebra, if a matrix $A$ corresponds to a linear map $\sigma:V\to W$ on suitable basis, then the inverse transpose $A^*$ of $A$ corresponds to the linear map $(\sigma^\vee)^{-1}:V^\vee\to W^\vee$.
Hence the dual bundle's fiber is dual to the original bundle.
\begin{sym}
    We use $\d{z}_\alpha^i\ (i=1,\cdots,n)$ to denote the dual basis of $\partial/\partial{z_\alpha^i}$
\end{sym}

We now give a brief treatment of differential forms.
\begin{defn}
    A \emph{differential form} of type $(p,q)$ (or $(p,q)$-form) on an open set $W$ is a differentiable section of $\left(\bigoplus_{i=1}^pT^*(M)\right)\oplus\left(\bigoplus_{j=1}^q\overline{T^*(M)}\right)$ over $W$ such that the fiber coordinate is skew-symmetric respect to indices.
    In general, we denote a $(p,q)$-form as
    \[\varphi(z)=\frac{1}{p!q!}\sum_{\stackrel{i_1,\cdots,i_p}{j_1,\cdots,j_q}}\varphi_{i_1\cdots i_pj_1\cdots j_q}(z)\d{z}^{i_1}\wedge\cdots\wedge\d{z}^{i_p}\wedge\d{z}^{\overline{j_1}}\wedge\cdots\wedge\d{z}^{\overline{j_q}}\]
    on a trivializing open set $U$, where $\d{z}^{\overline{j}}=\overline{\d{z}^j}$, and $\wedge$ is the wedge product.
\end{defn}

Similarly, we can define differential form on real differentiable manifolds.
\begin{defn}
    A \emph{differential form} of degree $p$ is a differentiable section of $\bigoplus_{i=1}^p\mathcal{T}(M)$ whose fiber coordinate is skew-symmetric.
    In general, a differential form of degree $p$ has form
    \[\varphi(x)=\sum\varphi_{i_1\cdots i_p}\d{x}^{i_0}\wedge\cdots\wedge\d{x}^{i_p}\]
    on a trivializing open set $U$.
\end{defn}

We define wedge product and exterior differential of differential forms.

\begin{defn}
    If
    \begin{gather*}
        \varphi=\frac{1}{p!}\sum\varphi_{i_1\cdots i_p}\d{x}^{i_1}\wedge\cdots\wedge\d{x}^{i_p}\\
        \psi=\frac{1}{q!}\sum\psi_{j_1\cdots j_q}\d{x}^{j_1}\wedge\cdots\wedge\d{x}^{j_q}\\
    \end{gather*}
    Then the \emph{wedge product} of $\varphi$ and $\psi$ is defined by
    \[\varphi\wedge\psi:=\frac{1}{p!q!}\sum\varphi_{i_1\cdots i_p}\psi_{j_1\cdots j_q}\d{x}^{i_1}\wedge\cdots\wedge\d{x}^{i_p}\wedge\d{x}^{j_1}\wedge\cdots\wedge\d{x}^{j_q}\]
    and the \emph{exterior differential} of $\varphi$ is defined by
    \begin{align*}
        \d{\varphi}&:=\frac{1}{p!}\sum\frac{\partial\varphi_{i_1\cdots i_p}}{\partial{x}^i}\d{x}^i\wedge\d{x}^{i_1}\wedge\cdots\wedge\d{x}^{i_p}\\
        &=\frac{1}{(p+1)!}\sum\phi_{i_0i_1\cdots i_p}\d{x}^{i_0}\wedge\d{x}^{i_1}\wedge\cdots\wedge\d{x}^{i_p}
    \end{align*}
    where
    \[\phi_{i_0i_1\cdots i_p}=\sum_{j=0}^p(-1)^j\frac{\partial\varphi_{i_0\cdots\widehat{i_j}\cdots i_p}}{\partial{x}^j}\]
\end{defn}

One can check that wedge product and exterior differential are well-defined, that is, do not depend on the choice of coordinate chart.
See~\cite[Section~9.3]{Tu11}.

We list two easy properties of exterior differential here.

\begin{prop}
    If $\varphi$ is a $p$-form and $\psi$ is a $q$-form, then
    \begin{enumerate}[\rm(1)]
        \item $\d{\d{\varphi}}=0$;
        \item $\d{(\varphi\wedge\psi)}=\d{\varphi}\wedge\psi+(-1)^p\varphi\wedge\d{\psi}$.
    \end{enumerate}
\end{prop}

\begin{sym}
    We use $A^p$ to denote the sheaf of differential $p$-forms on manifold $M$.
\end{sym}

\begin{thm}[Poincar\'{e}'s lemma]
    Suppose a $p$-form $\varphi$, $p\geq 1$, satisfies $\d\varphi=0$ on a star-shaped domain $W$ of $0$, then there exists a $(p-1)$-form $\psi$ on $W$ such that $\d\psi=\varphi$.
\end{thm}
\begin{proof}
    We construct a chain homotopy between $\mathrm{id}$ and $0$ on $A^\bullet$ of $W$.
    Denote $I^p=(i_1,\cdots,i_p)$ and $I^{p+1}=(i_0,i_1,\cdots,i_p)$ for short.
    Let
    \[\varphi=\frac{1}{p!}\sum_{I^p}\varphi_{I^p}\d{x}^{I^p}\]
    Then $h:A^p\to A^{p-1}$ is defined by
    \[h(\varphi)(x):=\frac{1}{p!}\sum_{I^p}\sum_{j=1}^p(-1)^{j-1}\left(\int_0^1t^{p-1}\varphi_{I^p}(tx)\d{t}\right)x^{i_j}\d{x}^{I^p\setminus(i_j)}\]
    Where $I^p\setminus(i_j)=(i_1,\cdots,\widehat{i_j},\cdots,i_p)$.
    Since $W$ is star-shaped, $h$ is well-defined.
    Straightforward computation shows
    \begin{align*}
        \d{h(\varphi)}(x)=&\frac{1}{p!}\sum_{I^p}\sum_{j=1}^p(-1)^{j-1}\sum_{k=1}^n\frac{\partial}{\partial{x}^k}\left(\int_0^1\varphi_{i^p}(tx)t^{p-1}\d{t}\right)x^{i_j}\d{x}^k\wedge\d{x}^{I^p\setminus(i_j)}\\
        =&\frac{1}{p!}\sum_{I^p}p\left(\int_0^1\varphi_{I^p}(tx)t^{p-1}\d{t}\right)\d{x}^{I^p}\\
        &+\frac{1}{p!}\sum_{I^p}\sum_{j=1}^p\sum_{k=1}^n(-1)^{j-1}\left(\int_0^1\frac{\partial{\varphi_{I^p}}}{\partial{x}^k}(tx)t^{p-1\d{t}}\right)x^{i_j}\d{x}^k\wedge\d{x}^{I^p\setminus(i_j)}
    \end{align*}
    and
    \begin{align*}
        h(\d\varphi)(x)=&h\left(\frac{1}{p!}\sum_{I^p}\sum_{k=1}^n\frac{\partial{\varphi_{I^p}}}{\partial{x}^k}\d{x}^k\wedge\d{x}^{I^p}\right)\\
        =&\frac{1}{p!}\sum_{I^p}\sum_{k=1}^n\left(\int_0^1\frac{\partial{\varphi_{I^p}}}{\partial{x}^k}(tx)t^p\right)x^k\d{x}^{I^p}\\
        &+\frac{1}{p!}\sum_{I^p}\sum_{k=1}^n\sum_{j=2}^{p+1}(-1)^{j-1}\left(\int_0^1\frac{\partial{\varphi_{I^p}}}{\partial{x}^k}(tx)t^p\d{t}\right)x^{i_{j-1}}\d{x}^k\wedge\d{x}^{I^p\setminus(i_{j-1})}\\
        =&\frac{1}{p!}\sum_{I^p}\sum_{k=1}^n\left(\int_0^1\frac{\partial{\varphi_{I^p}}}{\partial{x}^k}(tx)t^p\right)x^k\d{x}^{I^p}\\
        &+\frac{1}{p!}\sum_{I^p}\sum_{k=1}^n\sum_{j=1}^{p}(-1)^{j}\left(\int_0^1\frac{\partial{\varphi_{I^p}}}{\partial{x}^k}(tx)t^p\d{t}\right)x^{i_{j}}\d{x}^k\wedge\d{x}^{I^p\setminus(i_{j})}
    \end{align*}
    Adding $\d{h(\varphi)}$ and $h(\d{\varphi})$, the triple sums cancel, and we obtain
    \begin{align*}
        \d{h(\varphi)}(x)+h(\d{\varphi})(x)=&\frac{1}{p!}\sum_{I^p}\left(\int_0^1\left(pt^{p-1}\varphi_{I^p}(tx)+t^p\sum_{k=1}^nx^k\frac{\partial\varphi_{I^p}}{\partial{x}^k}(tx)\right)\d{t}\right)\d{x}^{I^p}\\
        =&\frac{1}{p!}\left(\sum_{I^p}\int_0^1\frac{\d{}}{\d{t}}\left(t^p\varphi_{I^p}(tx)\right)\d{t}\right)\d{x}^{I^p}\\
        =&\frac{1}{p!}\sum_{I^p}\varphi_{I^p}(x)\d{x}^{I^p}\\
        =&\varphi(x)
    \end{align*}
    Hence on $W$ $\mathrm{id}$ is chain homotopic to $0$, that is, for any $\varphi\in A^p$ with $\d\varphi=0$ there is a $\psi\in A^{p-1}$ such that $\d{\psi}=\varphi$ for $p\geq 1$. 
\end{proof}

\begin{defn}
    By a \emph{fine resolution} of sheaf $\mathcal{F}$ we mean an exact sequence of sheaves
    \[0\to\mathcal{F}\to\mathcal{A}^0\to\mathcal{A}^1\to\cdots\]
    such that each $\mathcal{A}^p$ is fine.
\end{defn}

\begin{prop}\label{fine resolution of C}
    The sequence
    \[0\to\mathbb{C}\to A^0\xrightarrow{\d^0{}} A^1\xrightarrow{\d^1{}} A^2\xrightarrow{\d^2{}}\cdots\]
    gives a fine resolution of $\mathbb{C}$.
\end{prop}
\begin{proof}
    First we notice given an open cover $\underline{U}$, each $A^p$ admits a partition of unity subordinate to $\underline{U}$, hence $A^p$ is fine.
    Next we notice Poincar\'{e}'s lemma shows $\im{\d^{p-1}{}}=\ker{\d^p{}}$ on each stalk, hence the sequence is exact.
\end{proof}

Finally we prove the important de Rham's theorem.
In fact, we have an enhanced conclusion.
\begin{thm}\label{fine de rham}
    Let
    \[0\to\mathcal{F}\to\mathcal{A}^0\xrightarrow{h}\mathcal{A}^1\xrightarrow{h}\mathcal{A}^2\xrightarrow{h}\cdots\]
    be a fine resolution of sheaf $\mathcal{F}$, then we have $hH^0(X,\mathcal{A}^{q-1})\subset H^0(X,h\mathcal{A}^{q-1})$, and
    \[H^q(X,\mathcal{F})\cong\frac{H^0(X,h\mathcal{A}^{q-1})}{hH^0(X,\mathcal{A}^{q-1})}\]
    for $q\geq 1$.
\end{thm}
\begin{proof}
    Since $h^{q-1}H^0(X,\mathcal{A}^{q-1})=\preim(h^{q-1})(X)$, which is naturally embedded into $\im(h^{q-1})(X)=H^0(X,h^{q-1}\mathcal{A}^{q-1})$ by Lemma~\ref{inj of sheafification map}, the inclusion holds.
    For the rest of the theorem, we use Theorem~\ref{exact cohomology groups}~repeatedly.
    Decompose the resolution into short exact sequence, we have for $p>0$
    \[0\to h\mathcal{A}^{p-1}\to\mathcal{A}^p\to h\mathcal{A}^p\to 0\]
    is exact.
    Then we have the long exact sequence as in Theorem~\ref{coho short exact}.
    Since $\mathcal{A}^p$ is fine, $H^q(\mathcal{A}^p)=0$ for any $q>0$, we have
    \begin{equation}
        H^q(X,h\mathcal{A}^p)\cong H^{q+1}(X,h\mathcal{A}^{p-1})\quad \forall p,q>0\label{de rham 1}
    \end{equation}
    For $q=0$, we have exact sequence
    \[0\to H^0(X,h\mathcal{A}^{p-1})\to H^0(X,\mathcal{A}^p)\to H^0(X,h\mathcal{A}^p)\to H^1(X,h\mathcal{A}^{p-1})\to 0\]
    Hence we have
    \begin{equation}
        H^1(X,h\mathcal{A}^{p-1})\cong\frac{H^0(X,h\mathcal{A}^p)}{hH^0(X,\mathcal{A}^p)}\label{de rham 2}
    \end{equation}
    For $p=0$, we have the short exact sequence
    \[0\to\mathcal{F}\to\mathcal{A}^0\to h\mathcal{A}^0\to 0\]
    Then we have the long exact sequence as in Theorem~\ref{coho short exact}.
    Since $\mathcal{A}^0$ is fine, $H^q(\mathcal{A}^0)=0$ for any $q>0$, then we have an exact sequence
    \[0\to H^0(X,\mathcal{F})\to H^0(X,\mathcal{A}^0)\to H^0(X,h\mathcal{A}^0)\to H^1(X,\mathcal{F})\to 0\]
    This implies
    \begin{equation}
        H^1(X,\mathcal{F})\cong\frac{H^0(X,h\mathcal{A}^0)}{hH^0(X,\mathcal{A}^0)}\label{de rham 3}
    \end{equation}
    And for $q>0$ we have
    \[0\to H^q(X,h\mathcal{A}^0)\to H^{q+1}(X,\mathcal{F})\to 0\]
    That is, $H^{q+1}(X,\mathcal{F})\cong H^q(X,h\mathcal{A}^0)$.
    Using~\eqref{de rham 1}~and~\eqref{de rham 2}, we have
    \begin{equation}
        \begin{aligned}
            H^q(X,\mathcal{F})&\cong H^{q-1}(X,h\mathcal{A}^0)\\
            &\cong\cdots\\
            &\cong H^1(X,h\mathcal{A}^{q-2})\\
            &\cong\frac{H^0(X,h\mathcal{A}^{q-1})}{hH^0(X,\mathcal{A}^{q-1})}
        \end{aligned}\label{de rham 4}
    \end{equation}
    Combining~\eqref{de rham 3}~and~\eqref{de rham 4}~we reach the conclusion.
\end{proof}

\begin{cor}[de Rham's Theorem]\label{de rham}
    We have the isomorphism
    \[H^q(X,\mathbb{C})\cong\frac{H^0(X,\d{A^{q-1}})}{\d{H^0(X,A^{q-1})}}\quad\forall q>0\]
\end{cor}
\begin{proof}
    Combine Theorem~\ref{fine de rham}~and Proposition~\ref{fine resolution of C}.
\end{proof}

\begin{rem}
    Generally speaking, we often define de Rham cohomology group of complex coefficients as the cohomology of the cochain complex
    \[0\to\mathbb{C}(X)\to A^0(X)\xrightarrow{\d{}}A^1(X)\xrightarrow{\d{}}A^2(X)\xrightarrow{\d{}}\cdots\]
    in algebraic topology.
    Denote the $q$th de Rham cohomology group by $H^q_{\mathrm{DR}}(X,\mathbb{C})$, then traditionally de Rham's theorem reads $\check{H}^q(X,\mathbb{C})\cong H^q_{\mathrm{DR}}(X,\mathbb{C})$.
    This is equivalent to the form we state in Corollary~\ref{de rham}.
\end{rem}

\section{Dolbeault's Lemmas}