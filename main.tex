\documentclass[a4paper]{book}

% \usepackage{geometry}
\usepackage{amsmath}
\usepackage{amssymb}
\usepackage{amsthm}
\usepackage{times}
\usepackage{mathptmx}
\usepackage{tikz-cd}
\usepackage{hyperref}
\usepackage{fixdif}

\bibliographystyle{alpha}

\theoremstyle{definition}
\newtheorem{defn}{Definition}[section]
\newtheorem{eg}[defn]{Example}
\newtheorem{cons}[defn]{Construction}
\newtheorem{sym}[defn]{Notation}
\theoremstyle{plain}
\newtheorem{thm}[defn]{Theorem}
\newtheorem{prop}[defn]{Proposition}
\newtheorem{cor}[defn]{Corollary}
\newtheorem{lem}[defn]{Lemma}
\theoremstyle{remark}
\newtheorem{rem}[defn]{Remark}

\title{A Note on Complex Manifolds}
\author{Zeng Mengchen}
\date{Last compile: \today}

\begin{document}

\maketitle
\thispagestyle{empty}

\frontmatter

\tableofcontents
\newpage
\thispagestyle{empty}

\chapter{Preface}

This is a lecture note of a seminar on complex manifolds, by OM Society of School of Mathematical Scieces, Beijing Normal University.
We mainly follow \textsc{Kodaira} and \textsc{Morrow}'s classic~\cite{Kodaira06}.
We shall cover the part of complex manifolds, sheaf cohomology and geometry of complex manifolds.
Deformation theory will be skipped.
Numbering of sections will not follow the textbook, but for some important theorems we shall give the name or original numbering on the textbook.

This note is unfinished and will update continuously, it will be post on GitHub.
The repository name is \href{https://github.com/matthewzenm/complex-manifolds-seminar}{matthewzenm/complex-manifolds-seminar}.

\begin{flushright}
    Mengchen M.\ Zeng

    Aug 24, 2023
\end{flushright}

\newpage
\thispagestyle{empty}

\mainmatter

\chapter{Complex Manifolds}

In this chapter we introduce the elements of several complex variables and the notion of complex manifolds.
We also provide some examples of complex ma\-nifolds.

\section{Holomorphic maps}

\begin{defn}
    A complex valued function $f(z)$ on a connected open subset $W\subset\mathbb{C}^n$ is called \emph{holomorphic}, if for each $a=(a_1,\cdots,a_n)\in W$, $f(z)$ can be expanded as a convergent power series
    \[f(z)=\sum_{k_1\geq 0,\cdots,k_n\geq 0}c_{k_1\cdots k_n}(z_1-a_1)^{k_1}\cdots(z_n-a_n)^{k_n}\]
    in some neighborhood of $a$.
\end{defn}

From now on we shall use \emph{domain} to denote a connected open set.

\begin{prop}
    If $p(z)=\sum c_{k_1\cdots k_n}(z_1-a_1)^{k_1}\cdots(z_n-a_n)^{k_n}$ converges at $z=w$, then $p(z)$ converges for every $z$ with $|z_k-a_k|<|w_k-a_k|,\ k=1,\cdots,n$.
\end{prop}
\begin{proof}
    Trivial.
\end{proof}

\begin{defn}
    The neighborhood above is called a \emph{polydisc} or \emph{polycylinder}, and denoted by $P(a,r)=\{z\in\mathbb{C}^n:\ |z-a|<r\}$.
\end{defn}

A complex valued function of $n$ complex variables can be seen as a function of $2n$ real variables, thus we have the following definition.
\begin{defn}
    A complex valued function of $n$ complex variables is \emph{continuous} or \emph{differentiable}, if it is continuous or differentiable as a function of $2n$ real variables.
\end{defn}

We have

\begin{thm}[Osgood]\label{osgood}
    Let $f(z_1,\cdots,z_n)$ be a continuous function on the domain $W\subset\mathbb{C}^n$, if $f$ is holomorphic with respected to each $z_k$ and other $z_i$'s fixed, then $f$ is holomorphic on $W$.
\end{thm}
\begin{proof}
    Let $a\in W$ lies in the polydisc $\overline{P(a,r)}\subset W$, we use Cauchy's integral formula iteratively:
    \begin{gather*}
        f(z_1,z_2,\cdots,z_n)=\frac{1}{2\pi\sqrt{-1}}\int_{|z_1-a_1|=r_1}\frac{f(w_1,z_2,\cdots,z_n)}{w_1-z_1}\d{w_1}\\
        f(w_1,z_2,\cdots,z_n)=\frac{1}{2\pi\sqrt{-1}}\int_{|z_2-a_2|=r_2}\frac{f(w_1,w_2,\cdots,z_n)}{w_2-z_2}\d{w_2}\\
        \cdots
    \end{gather*}
    Substituting, we have
    \[\left(\frac{1}{2\pi\sqrt{-1}}\right)^n\idotsint_{\partial P(a,r)}\frac{f(w_1,\cdots,w_n)}{(w_1-z_1)\cdots(w_n-z_n)}\d{w_1}\cdots\d{w_n}\]
    Since
    \[\left|\frac{z_k-a_k}{w_k-a_k}\right|<1\]
    The series
    \begin{align*}
        \frac{1}{w_k-z_k}&=\frac{1}{(w_k-a_k)-(z_k-a_k)}=\frac{1}{w_k-a_k}\cdot\frac{1}{1-(z_k-a_k)/(w_k-a_k)}\\
        &=\frac{1}{w_k-a_k}\sum_{i=0}^\infty\left(\frac{z_k-a_k}{w_k-a_k}\right)^i
    \end{align*}
    converges absolutely in $P(a,r)$, hence integrate term by term we have
    \[f(z_1,\cdots,z_n)=\sum_{k_0\geq 0,\cdots,k_n\geq 0}c_{k_1\cdots k_n}(z_1-a_1)^{k_1}\cdots(z_n-a_n)^{k_n}\]
    where
    \[c_{k_1\cdots k_n}=\left(\frac{1}{2\pi\sqrt{-1}}\right)^{k_1+\cdots+k_n}\idotsint_{\partial P(a,r)}\frac{f(w_1,\cdots,w_n)\d{w_1}\cdots\d{w_n}}{(w_1-a_1)^{k_1+1}\cdots(w_n-a_n)^{k_n+1}}\]
    Let $|f|\leq M$ on $\overline{P(a,r)}$, then we have
    \[|c_{k_0\cdots k_n}|\leq\frac{M}{r_1^{k_1}\cdots r_n^{k_n}}\]
    and for $z\in P(a,r)$, we have $|(z_k-a_k)/r_k|<1$, then
    \begin{align*}
        \left|\sum c_{k_1\cdots k_n}(z_1-a_1)^{k_1}\cdots(z_n-a_n)^{k_n}\right|&\leq M\left|\sum\left(\frac{z_1-a_1}{r_1}\right)^{k_1}\cdots\left(\frac{z_n-a_n}{r_n}\right)^{k_n}\right|\\
        &=M\prod_{k=1}^n\left|\frac{1}{1-(z_k-a_k)/r_k}\right|
    \end{align*}
    This shows the expansion is convergent for $z\in P(a,r)$.
    Since $a$ is arbitrary, $f$ is holomorphic.
\end{proof}

We now introduce the Cauchy--Riemann equations.

\begin{sym}
    Let $f$ be a differentiable function on a domain $W\subset\mathbb{C}^n$, denote
    \begin{gather}
        \frac{\partial}{\partial z_k}=\frac{1}{2}\left(\frac{\partial}{\partial x_k}-\sqrt{-1}\frac{\partial}{\partial y_k}\right)\\
        \frac{\partial}{\partial\overline{z_k}}=\frac{1}{2}\left(\frac{\partial}{\partial x_k}+\sqrt{-1}\frac{\partial}{\partial y_k}\right)
    \end{gather}
    for $z_k=x_k+\sqrt{-1}y_k$ and $1\leq k\leq n$.
\end{sym}

\begin{thm}
    Let $f$ be a (continuously) differentiable function on the domain $W\subset\mathbb{C}^n$, then $f$ is holomorphic on $W$ if and only if $\partial{f}/\partial{\overline{z_k}}=0$ for $k=1,\cdots,n$.
\end{thm}
\begin{proof}
    This is a corollary of Theorem~\ref{osgood} and classical results in complex analysis in one variable.
\end{proof}

\begin{prop}[Chain rule]
    Suppose $f(w_1,\cdots,w_m)$ and $g_k(z),\ k=1,\cdots,m$ are differentiable, and the domain of $f$ contains the range of $g=(g_1,\cdots,g_m)$, then $f\circ g$ is differentiable, and if $w_m=g_m(z)$, then
    \begin{gather*}
        \frac{\partial f(g(z)))}{\partial z_k}=\sum_{i=1}^m\left(\frac{\partial f(w)}{\partial w_i}\cdot\frac{\partial w_i}{\partial z_k}+\frac{\partial f(w)}{\partial\overline{w_i}}\cdot\frac{\partial\overline{w_i}}{\partial z_k}\right)\\
        \frac{\partial f(g(z)))}{\partial\overline{z_k}}=\sum_{i=1}^m\left(\frac{\partial f(w)}{\partial w_i}\cdot\frac{\partial w_i}{\partial\overline{z_k}}+\frac{\partial f(w)}{\partial\overline{w_i}}\cdot\frac{\partial\overline{w_i}}{\partial\overline{z_k}}\right)
    \end{gather*}
\end{prop}
\begin{proof}
    Direct calculation verifies the proposition.
\end{proof}

\begin{cor}
    If $f(w)$ is holomorphic in $w=(w_1,\cdots,w_m)$ and $g_k(z),\ k=1,\cdots,m$ are holomorphic in $z$, then $f\circ g$ is holomorphic in $z$.
\end{cor}

\begin{cor}
    The set $\Gamma(\Omega,\mathcal{O}_{\mathbb{C}^n})$ of holomorphic functions on open set $\Omega$ forms a ring. (We use sheaf notation before we introduce what is a sheaf.)
\end{cor}

\begin{defn}
    A map $f(z)=(f_1(z),\cdots,f_m(z))$ from $\mathbb{C}^n$ to $\mathbb{C}^m$ is a \emph{holomorphic map} if each $f_k(z)$ is holomorphic, $k=1,\cdots,m$.
\end{defn}

\backmatter

\bibliography{biblio}

\end{document}